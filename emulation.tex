\chapter{Emulation Systems}
\label{chap:emulation}

To analyze a system, we must effectively represent its behavior for some case, or set of cases, of interest.
This dissertation is concerned with the emulation of systems or the effective duplication of a system for the purpose of identifying flaws, particularly with respect to security.
In Chapter~\ref{chap:rehost}, we will consider the process of rehosting a system to a different domain, in the context of hardware and instruction set emulators, which often allow complex embedded and cyberphysical systems to be analyzed on consumer desktop computers, e.g. laptops, and will consider dynamic analysis, a form of simulation which involves instrumentation of the system under analysis.
Chapter~\ref{chap:integreat} and Chapter~\ref{chap:info} will discuss consider high-level emulators, copies of a system in a more abstract domain, and function extraction, strategies allowing researchers to dissect complex hardware, firmware, and software systems into testable subcomponents.
Both Chapter~\ref{chap:rehost} and Chapter~\ref{chap:rehost} consider the use of symbolic execution and abstract interpretation as a mechanism for emulation, as these techniques explore multiple execution paths and thus implicitly emulate any single, particular execution trace as long as their lifting, semantics, and search strategies are able to identify it.
Every one of these discussed formalisms is well-known and each has several systems that implement the technique or strategy for a variety of real-world use cases~\cite{}.
This dissertation may be the first to organize these approaches as forms of emulation---detailed literature reviews are included alongside Chapters~\ref{chap:rehost},\ \ref{chap:info}, and~\ref{chap:integreat}.

In this chapter, we will introduce four different forms of emulation used in the analysis of complex systems.
For each form, we will identify contemporary system that implement the approach, define its nature and limitations, and provide examples.

\section{Hardware Emulators and Instruction Set Simulators}
\label{sec:hardemu}

\section{Dynamic Analysis}

Last, the perfect hardware emulator is a non-intrusive observation of the target system.
An ideal dynamic analysis framework provides access to the original system's behavior without modifying the system.
This is not always possible, but it is a goal to strive for.

\subsection{Binary Rewriting}

\subsection{Fuzzing}

\subsection{Debuggers}


\section{Functional and High Level Emulators}

\subsection{Function Extraction and Summarization}

\subsection{Environmental Modeling}

\section{Symbolic Executors}

\subsection{Taint Tracking and Data Flow Analysis}
