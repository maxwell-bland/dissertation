% {{{ packages

\usepackage{blindtext}

% math
\usepackage{amsmath}
\usepackage{amsthm}
\usepackage{stmaryrd}
\usepackage{cases}

% pretty links
\usepackage{xparse}
\usepackage{hyperref}
\usepackage[nameinlink]{cleveref}
\hypersetup{
    colorlinks=true,
    urlcolor=blue,
    citecolor=black,
    linkcolor=black
}

% better environments
\usepackage[shortlabels]{enumitem}
\usepackage{booktabs}
\usepackage{caption}

% graphics
\usepackage{tikz}
\usetikzlibrary{calc}

% fancier font
\usepackage[sc]{mathpazo}
% better typography
\usepackage[activate={true,nocompatibility}, % activate protrusion and font expansion
            final,              % enable microtype, use draft to disable
            tracking=true,
            kerning=true,       % optimise interactions between characters
            spacing=true,       % more uniform spacing between words
            factor=1100,        % more protrusion
            stretch=10,         % smaller values (default 20, 20) to avoid blurring
            shrink=10]{microtype}
\SetTracking{encoding={*}, shape=sc}{40}

% }}}

% {{{ formatting

% enable section numbering only for the first three header levels
\setcounter{secnumdepth}{2}

% caption format
% NOTE: set format=plain to remove caption indentation
\captionsetup{
    format=hang
}

% allow subsubsections in the TOC, if there are any
% NOTE: if you got to subsubsections, you probably have too many!
\setcounter{tocdepth}{\subsubsectiontocdepth}
\setcounter{secnumdepth}{\subsubsectionnumdepth}

% NOTE: if there are many levels of indentation int the TOC, making it flat
% might be a good option with the following options
% \KOMAoption{toc}{indenttextentries,flat}

% https://marketing.illinois.edu/visual-identity/color
\definecolor{IlliniOrange}{RGB}{255, 95, 5}
\definecolor{IlliniAltgeld}{RGB}{200, 65, 19}
\definecolor{IlliniBlue}{RGB}{19, 41, 75}

\definecolor{IlliniAlmaMater}{RGB}{30, 56, 119}
\definecolor{IlliniIndustrialBlue}{RGB}{29, 88, 167}
\definecolor{IlliniArchesBlue}{RGB}{0, 159, 212}
\definecolor{IlliniCloud}{RGB}{248, 250, 252}
\definecolor{IlliniHeritageOrange}{RGB}{245, 130, 30}

% }}}

% {{{ commands

\NewDocumentCommand \dx { O{x} } {\,\mathrm{d} #1}
\NewDocumentCommand \vect { m } { \mathbold{#1} }
\NewDocumentCommand \od { m m } { \dfrac{\mathrm{d} #1}{\mathrm{d} #2} }
\NewDocumentCommand \pd { m m } { \dfrac{\partial #1}{\partial #2} }

% jump notation
\NewDocumentCommand \jump { sm } {
    \IfBooleanTF#1
    {\left\llbracket #2 \right\rrbracket}
    {\llbracket #2 \rrbracket}
}
% average notation
\NewDocumentCommand \avg { sm } {
    \IfBooleanTF#1
    {\left\langle #2 \right\rangle}
    {\langle #2 \rangle}
}
% inner product
\NewDocumentCommand \ip { m } { \avg{ #1 } }

\DeclareMathOperator{\tr}{tr}
\DeclareMathOperator{\sech}{sech}
\DeclareMathOperator{\argmin}{\operatorname{arg}\,\operatorname{min}}

% }}}

% {{{ environments

\NewDocumentCommand \newtheoremin { m m m } {
    \newtheorem{#1}{#2}
    \numberwithin{#1}{#3}
}
\newtheoremin{example}{Example}{section}
\newtheoremin{remark}{Remark}{section}
\newtheoremin{definition}{Definition}{section}
\newtheoremin{proposition}{Proposition}{section}
\newtheoremin{lemma}{Lemma}{section}
\newtheoremin{theorem}{Theorem}{section}

% }}}
